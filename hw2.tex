{\rtf1\ansi\ansicpg1252\cocoartf1561\cocoasubrtf600
{\fonttbl\f0\fswiss\fcharset0 Helvetica;}
{\colortbl;\red255\green255\blue255;}
{\*\expandedcolortbl;;}
\margl1440\margr1440\vieww10800\viewh8400\viewkind0
\pard\tx720\tx1440\tx2160\tx2880\tx3600\tx4320\tx5040\tx5760\tx6480\tx7200\tx7920\tx8640\pardirnatural\partightenfactor0

\f0\fs24 \cf0 \\documentclass\{article\}\
\\usepackage[utf8]\{inputenc\}\
\\usepackage\{amsmath\}\
\\usepackage\{graphicx\}\
\\title\{ee364a HW2\}\
\\author\{ilangold \}\
\\date\{September 2019\}\
\
\\begin\{document\}\
\
\\maketitle\
\
\\section*\{3.2\}\
There are roughly three corners - in the top right corner, the function increases rapidly to the second level curve and then more slowly to the third curve - this represents a sort of "kink" in the function that rules out convexity.  Clearly, the function could not be concave as the superlevel sets according to this are not convex, thus ruling our quasiconcavity as well. It could be quasi-convex though.\
\
These definitely could not be quasi-convex or convex as they could not be convex sublevel sets.  They could however be superlevel sets of quasiconcave/concave functions.\
\
\\section*\{3.15\}\
\\subsection*\{a\}\
We can use l'hopital's rule - we know that\
\\begin\{center\}\
    $lim_\{x \\rightarrow 0\} \\frac\{x^\\alpha - 1\}\{\\alpha\} = \\frac\{\\frac\{d\}\{d \\alpha\}|_\{0\}(x^\\alpha - 1)\}\{\\frac\{d\}\{d \\alpha\}|_\{0\}(\\alpha)\} = log(x) * x^\{0\} = log(x)$ \
\\end\{center\}\
\\subsection*\{b\}\
Clearly all satisfy $u_\{\\alpha\}(1) = 0$.  To see they are concave, we take two consecutive derivatives and see that $\\frac\{d^2\}\{d \\alpha^2\} \\frac\{x^\\alpha - 1\}\{\\alpha\} = \\alpha(\\alpha - 1) \\frac\{x^\\alpha - 2\}\{\\alpha\}$ which is clearly always negative since $\\alpha - 1$ is always negative and the other terms must be positive.  We then get for free that the function is monotone increasing as the first derivative is just $\\alpha \\frac\{x^\\alpha - 1\}\{\\alpha\}$ which must be positive.\
\\section*\{3.16\}\
\\subsection*\{a\}\
Clearly this function is convex as it is simply the exponential shifted down by one\
\\subsection*\{b\}\
This function is quasiconcanve as we showed on a previous homework problem that its superlevel sets are all convex (thus its not quasiconvex).  It is not convex or concave though, as we can show that it's hessian is neither poisitive semidefinite nor positive negative semidefinite.\
\\begin\{center\}\
\\[\
\
\\begin\{bmatrix\}\
    0       & 1 \\\\\
    1       & 0 \
\\end\{bmatrix\}\
\\]\
\\end\{center\}\
\
\\subsection*\{c\}\
This function is quasi-convex as it is the reciprocal relationship to above (again, see the last homework).  This function, though, has hessian\
\\begin\{center\}\
\\[\
\\begin\{bmatrix\}\
    $$\\frac\{2\}\{x_2x_1^3\}$$       &  $$\\frac\{1\}\{x_2^2x_1^2\}$$ \\\\\
    $$\\frac\{1\}\{x_2^2x_1^2\}$$       & $$\\frac\{2\}\{x_2^3x_1\}$$  \
\\end\{bmatrix\}\
\\]\
\\end\{center\}\
Which is certainly positive definite and thus the function is convex (and not concave).\
\
\\subsection*\{d\}\
This function has neither negative nor positive definite hessian matrix:\
\\begin\{center\}\
\\[\
\\begin\{bmatrix\}\
    $$0$$       &  $$\\frac\{-1\}\{x_1^2\}$$ \\\\\
    $$\\frac\{-1\}\{x_1^2\}$$       & $$\\frac\{x_1\}\{x_2^3\}$$  \
\\end\{bmatrix\}\
\\]\
\\end\{center\}\
\
However, it's sub and super level sets are both convex so it is both quasi convex and quasiconcave.\
\\newpage\
\
\\section*\{4.1\}\
The feasible set is above.\
\\begin\{figure\}\
    \\centering\
    \\includegraphics[scale=.3]\{feasible_41.png\}\
    \\caption\{Feasible Region\}\
    \\label\{fig:my_label\}\
\\end\{figure\}\
\\subsection*\{a\}\
This function achieves no minimum globally but over the feasible set, it is minimized at $(.4, .2)$ and has minimum $.6$.\
\\subsection*\{b\}\
This function has no optimal set and no optimal value as it is unbounded below as x and y go to infinity in the feasible set.\
\\subsection*\{c\}\
 This is minimized wherever $x_1 = 0$ in the feasible set.\
 \\subsection*\{d\}\
 This must achieve its optimal value at $(\\frac\{1\}\{3\}, \\frac\{1\}\{3\})$ as it lies on the red line and decreasing either would increase the other. \
 \\subsection*\{e\}\
 The function's minimum is $\\frac\{1\}\{2\}$ achieved at $(.5, \\frac\{1\}\{6\})$\
 \\section*\{A2.2\}\
 Since convexity can be determined by looking at a function restricted to any single line (i.e $n=1$), we know that any of these conditions work for $n=1$ for all $k$.  Note that the last condition simply says that $g_i$ is both convex and concave so it doesn't matter what $h$ does in that coordinate.  Looking at the expression for the second derivative, we see however that only one of the conditions need to hold for each $i$ in order for the positive-definiteness to hold as the dot product goes element wise i.e if $h$ non-increasing in $i$ and $g_i$ concave, then $\\frac\{\\partial h(g(x))\}\{\\partial g_i\}  * g_i$ is positive (and a similar argument for the quadratic term):\
 \
 \\begin\{center\}\
     $f^\{''\}(x) = g^\{'\}(x) \\nabla^2 h(g(x)) g^\{'\}(x) + \\nabla h(g(x))^T g^\{''\}(x)$\
 \\end\{center\}\
  \\section*\{A2.34\}\
  \\subsection*\{a\}\
  This set is certainly convex - simply take $P$ and $Q$ and note: \
  \\begin\{center\}\
      $x^t (\\theta P + (1-\\theta) Q) x = \\theta x^t P x + (1-\\theta) x^T Q x$\
  \\end\{center\}\
  which is certainly greater than 0 as a non-negative scalar times something non-negative is non-negative still.\
  \\subsection*\{b\}\
  This problem can be reduced to the intersection of an infinite number of convex sets (indexed over t) defined by a system of two linear inequalities and an equality:\
\\[\
\\begin\{bmatrix\}\
    $$ c_0 $$   \\\\\
    $$ c_1 $$   \\\\\
    $$ c_2 $$\
\\end\{bmatrix\} ^ T\
\\begin\{bmatrix\}\
    $$ 1 $$   \\\\\
    $$ t $$   \\\\\
    $$ t^2 $$\
\\end\{bmatrix\} \
< 1\\] \\[\
\\begin\{bmatrix\}\
    $$ c_0 $$   \\\\\
    $$ c_1 $$   \\\\\
    $$ c_2 $$\
\\end\{bmatrix\} ^ T\
\\begin\{bmatrix\}\
    $$ 1 $$   \\\\\
    $$ t $$   \\\\\
    $$ t^2 $$\
\\end\{bmatrix\} \
> -1\\] \
\
\\[\
\\begin\{bmatrix\}\
    $$ c_0 $$   \\\\\
    $$ c_1 $$   \\\\\
    $$ c_2 $$\
\\end\{bmatrix\} ^ T\
\\begin\{bmatrix\}\
    $$ 1 $$   \\\\\
    $$ 0 $$   \\\\\
    $$ 0 $$\
\\end\{bmatrix\} \
= 1\\] \
  \\subsection*\{c\}\
  This region is very much so convex as the intersection of three convex sets: we know that $cos(u+v) \\geq \\frac\{\\sqrt\{2\}\}\{2\}$ implies that $u, v$ must satisfy two linear inequalities: $\\frac\{-\\pi\}\{4\} \\leq u + v \\leq \\frac\{\\pi\}\{4\}$ and the second condition implies that they must lie inside a disc.  Thus the set is convex.\
  \\subsection*\{d\}\
  Since the inverse of a negative definite matrix is itself negative definite, we know that this set is empty and thus trivially convex.\
  \\section*\{A2.36\}\
  \\subsection*\{a\} This function is not DCP compliant since the square root function composed with squares is not.\
  \\subsection*\{b\} This function is DCP compliant and is concave.\
   \\subsection*\{c\} This function is DCP compliant and convex since it is the log of the sum of exponents. \
   \\section*\{A2.39\}\
   \\subsection*\{a\}\
   This set is convex as it can be converted into a linear inequality (note there is no hole at 0 as this is only defined on the positive orthant).\
   \\subsection*\{b\}\
   Same as above.\
   \\subsection*\{c\}\
   This set is not convex - the points $(4,.0001)$ and $(.0001,4)$ are in the set but $(2,2)$ is not.\
   \\subsection*\{d\}\
   This set is convex as it is the epigrah of the convex function $y = \\frac\{1\}\{x\}$ (which has positive second derivative over the non-negative reals).\
\\end\{document\}\
\
}